%%%% macros.tex by Stefano Gogioso
%%%% Version 12 Apr 2017 


%% Theorem environments - Comment out for certain journal submissions and for beamer 
	%% Counters and miscellaneous
  \newcounter{theoremUnified} % Unified coutner for all theorem environments
  \def\thetheoremUnified{\arabic{section}} % Needed to have counters going with sections
  \numberwithin{theoremUnified}{section} % Numbering within sections
  \numberwithin{theoremUnified}{section} % Equations are also numbered within sections

% Theorem Styles
%
  \newtheoremstyle{plainStyle} % Plain theorem style
  {2mm} % Space above
  {2mm} % Space below
  {} % Body font
  {} % Indent amount
  {\bfseries} % Theorem head font
  {.} % Punctuation after theorem head
  {.5em} % Space after theorem head
  {} % Theorem head spec (can be left empty, meaning `normal')

  \newtheoremstyle{italicStyle} % Italic theorem style
  {2mm} % Space above
  {2mm} % Space below
  {\itshape} % Body font
  {} % Indent amount
  {\bfseries} % Theorem head font
  {.} % Punctuation after theorem head
  {.5em} % Space after theorem head
  {} % Theorem head spec (can be left empty, meaning `normal')

% Define the environments
%
  \theoremstyle{plainStyle} % Environments in plain style
    \newtheorem{example}[theoremUnified]{Example}
      \AfterEndEnvironment{example}{\noindent\ignorespaces}
    \newtheorem{fact}[theoremUnified]{Fact}
      \AfterEndEnvironment{fact}{\noindent\ignorespaces}
    \newtheorem{examplehard}[theoremUnified]{Example*}
      \AfterEndEnvironment{examplehard}{\noindent\ignorespaces}
    \newtheorem{remark}[theoremUnified]{Remark}
      \AfterEndEnvironment{remark}{\noindent\ignorespaces}
    \newtheorem{remarkhard}[theoremUnified]{Remark*}
    \AfterEndEnvironment{remarkhard}{\noindent\ignorespaces}
%
  \theoremstyle{italicStyle} % Environments in italic style
    \newtheorem{definition}[theoremUnified]{Definition}
      \AfterEndEnvironment{definition}{\noindent\ignorespaces}
    \newtheorem{proposition}[theoremUnified]{Proposition}
      \AfterEndEnvironment{proposition}{\noindent\ignorespaces}
    \newtheorem{lemma}[theoremUnified]{Lemma}
      \AfterEndEnvironment{lemma}{\noindent\ignorespaces}
    \newtheorem{theorem}[theoremUnified]{Theorem}
      \AfterEndEnvironment{theorem}{\noindent\ignorespaces}
    \newtheorem{corollary}[theoremUnified]{Corollary}
      \AfterEndEnvironment{corollary}{\noindent\ignorespaces}

% Ubiquitous set names
\newcommand{\Naturals}{\mathbb{N}} % Set of natural numbers
\newcommand{\Integers}{\mathbb{Z}} % Set of interer numbers
\newcommand{\Bool}{\mathbb{B}} % Set of interer numbers

% Background for tikz images
\def\backgrnd{black!10}	% Background for Tikz pictures

% Basic Definitions
%
\newcommand{\Obj}[1]{\operatorname{Obj} \, #1} % Set of objects of category #1
\newcommand{\Mor}[1]{\operatorname{Mor} \, #1} % Set of objects of category #1
\newcommand{\GObj}[1]{\operatorname{GenObj} \, #1} % Set of objects of category #1
\newcommand{\GMor}[1]{\operatorname{GenMor} \, #1} % Set of objects of category #1

\newcommand{\Snark}[1]{\operatorname{Sn}(#1)} % Set of objects of category #1

\newcommand{\Homtotal}[1]{\operatorname{Hom}_{\,#1}} % Set of morphisms of category #1
\newcommand{\Hom}[3]{\operatorname{Hom}_{\,#1}\left[#2,#3\right]} % Set of morphisms of category #1 from object #2 to object #3
\newcommand{\Source}[2]{\operatorname{s}_{#1}(#2)} % Domain of function/morphism #1
\newcommand{\Target}[2]{\operatorname{t}_{#1}(#2)} % Domain of function/morphism #1
\newcommand{\Id}[1]{id_{#1}} % Identity morphism of object #1

% Generic names for categories
%
\newcommand{\CategoryA}{\mathcal{A}}
\newcommand{\CategoryB}{\mathcal{B}}
\newcommand{\CategoryC}{\mathcal{C}}
\newcommand{\CategoryD}{\mathcal{D}}
\newcommand{\CategoryE}{\mathcal{E}}

\newcommand{\Free}[1]{\mathfrak{F}(#1)}
\newcommand{\UnFree}[1]{\mathfrak{U}(#1)}

% Monoidal Categories
%
\newcommand{\Tensor}{\otimes} % Monoidal tensor
\newcommand{\TensorUnit}{I} % Monoidal tensor unit

% Logic
%
\newcommand{\Suchthat}[2]{\left\{#1 \: \middle\vert \: #2\right\}} % Set of elements #1 such that condition #2 holds 

% Category Names
\newcommand{\Bfun}{\Bool_\textbf{fun}} % Category of boolean functions
\newcommand{\Bcirc}{\Bool_\textbf{circ}} % Category of boolean circuits
\newcommand{\Bkp}{\Bool_\textbf{KP}} % Category of KP boolean circuits
\newcommand{\Bzkp}[1]{\Bool^{#1}_\textbf{ZKP}} % Category of KP boolean circuits

\newcommand{\Bpath}[1]{\Bool^{#1}_{\textbf{path}}} % Category of path proofs
\newcommand{\Count}{\textbf{Count}} % Category of KP boolean circuits
\newcommand{\Bgraph}[1]{\Bool^{#1}_{\Graph}} % Category of proof for a whole automaton
\newcommand{\Bsnark}{\Bool_\textbf{SNARK}} % Category of snark boolean circuits
\newcommand{\BsnarkSize}[1]{\Bool^{#1}_{\textbf{path}}} % Category of full snark proofs of some computational size

\newcommand{\Graph}{\textbf{Graph}} % Category of graphs
\newcommand{\Cat}{\textbf{Cat}} % Category of graphs

\newcommand{\NAND}{\ensuremath{\texttt{NAND}}\xspace}
\newcommand{\AND}{\ensuremath{\texttt{AND}}\xspace}
\newcommand{\OR}{\ensuremath{\texttt{OR}}\xspace}
\newcommand{\XOR}{\ensuremath{\texttt{XOR}}\xspace}
\newcommand{\COPY}{\ensuremath{\texttt{COPY}}\xspace}
\newcommand{\TRUE}{\ensuremath{\texttt{TRUE}}\xspace}
\newcommand{\FALSE}{\ensuremath{\texttt{FALSE}}\xspace}
\newcommand{\Zero}{\ensuremath{\textbf{0}}\xspace}


\newcommand{\NANDSym}{
  \scalebox{0.3}{
    \tikz[baseline=-10pt] \node[thick, american nand port] (char) {};
  }
}

\newcommand{\XORSymb}{
  \scalebox{0.3}{
    \tikz[baseline=-10pt] \node[thick, american xor port] (char) {};
  }
}

\newcommand{\ANDSym}{
  \scalebox{0.3}{
    \tikz[baseline=-10pt] \node[thick, american and port] (char) {};
  }
}

\newcommand{\ORSym}{
  \scalebox{0.3}{
    \tikz[baseline=-10pt] \node[thick, american or port] (char) {};
  }
}
\newcommand{\COPYSym}{
  \scalebox{0.3}{
    \tikz[baseline=-8pt]{
      \node[thick, draw, circle, radius=2pt] (copy) at (0,0) {};
      \node (in) at (-1,0) {};
      \node (out1) at (1,0.5) {};            
      \node (out2) at (1,-0.5) {};
      \draw[thick, -] (in.center) to (copy);
      \draw[thick, -, bend left] (copy) to (out1.center);
      \draw[thick, -, bend right] (copy) to (out2.center);
    } 
  }
}

\newcommand{\MATCHSym}{
  \scalebox{0.3}{
    \tikz[baseline=-8pt]{
      \node[thick, draw, fill=gray, circle, radius=2pt] (copy) at (0,0) {};
      \node (in) at (1,0) {};
      \node (out1) at (-1,0.5) {};            
      \node (out2) at (-1,-0.5) {};
      \draw[thick, -, dotted] (in.center) to (copy);
      \draw[thick, -, bend right] (copy) to (out1.center);
      \draw[thick, -, bend left] (copy) to (out2.center);
    } 
  }
}